% resume.tex
%
% (c) 2002 Matthew Boedicker <mboedick@mboedick.org> (original author) http://mboedick.org
% (c) 2003-2007 David J. Grant <davidgrant-at-gmail.com> http://www.davidgrant.ca
%
% This work is licensed under the Creative Commons Attribution-ShareAlike 3.0 Unported License. To view a copy of this license, visit http://creativecommons.org/licenses/by-sa/3.0/ or send a letter to Creative Commons, 171 Second Street, Suite 300, San Francisco, California, 94105, USA.

\documentclass[letterpaper,11pt]{article}

%-----------------------------------------------------------
%Margin setup

\setlength{\voffset}{0.1in}
\setlength{\paperwidth}{8.5in}
\setlength{\paperheight}{11in}
\setlength{\headheight}{0in}
\setlength{\headsep}{0in}
\setlength{\textheight}{10.5in}
%\setlength{\textheight}{9.5in}
\setlength{\topmargin}{-0.25in}
\setlength{\textwidth}{7in}
\setlength{\topskip}{0in}
\setlength{\oddsidemargin}{-0.25in}
\setlength{\evensidemargin}{-0.25in}
%\setlength{\oddsidemargin}{0.5in}
%\setlength{\evensidemargin}{0.5in}
%-----------------------------------------------------------
%\usepackage{fullpage}
\usepackage{shading}
%\textheight=9.0in
\pagestyle{empty}
\raggedbottom
\raggedright
\setlength{\tabcolsep}{0in}

%-----------------------------------------------------------
%Custom commands
\newcommand{\resitem}[1]{\item #1 \vspace{-2pt}}
\newcommand{\resheading}[1]{{\large \parashade[.9]{sharpcorners}{\textbf{#1 \vphantom{p\^{E}}}}}}
\newcommand{\ressubheading}[4]{
\begin{tabular*}{6.5in}{l@{\extracolsep{\fill}}r}
		\textbf{#1} & #2 \\
		\textit{#3} & \textit{#4} \\
\end{tabular*}\vspace{-6pt}}
%-----------------------------------------------------------


\begin{document}

\begin{tabular*}{7in}{l@{\extracolsep{\fill}}r}
    \textbf{\Large Markus Kusano}  & 571-212-9696\\
    13402 Alfred Mill Ct. &  mukusano@vt.edu \\
    Herndon, VA 20171 & http://github.com/markus-kusano \\
\end{tabular*}
\\

\vspace{0.1in}


\resheading{Objective}
Acceptance to Virginia Tech Computer Engineering PhD program

\resheading{Education}
\begin{itemize}
\item
	\ressubheading{Virginia Tech}{Blacksburg, VA}{Computer Engineering
        (GPA 3.9/4.0)}{Fall 2009 - Spring 2014}
	\begin{itemize}
            \resitem{Relevant courses: Embedded Systems Design, Applied Software Design, Digital Design II, Semiconductor Device Fundamentals, Electronics}
	\end{itemize}

\end{itemize}

\resheading{Awards}
\begin{itemize}
    \resitem{Listed on Dean's List every semester at Virginia Tech: Fall 2009,
        Spring 2010, Fall 2011, Spring 2012, Fall 2012, Spring 2013}
    \resitem{NSF REU Fellowship 2012-2013}
    \resitem{NSF travel support to SRI Summer School on Formal Techniques (May 20-24, 2013)}
\end{itemize}

\resheading{Publications}
\begin{itemize}
  \resitem{``CCmutator: A mutation generator for concurrency constructs in
  multithreaded C/C++ applications," Markus Kusano and Chao Wang. 28th IEEE/ACM
  International Conference on Automated Software Engineering. Palo Alto, CA,
  2013.}
  \resitem{``Hybrid Static and Dynamic Guided Concurrent Software Testing,''
  planned submission to ISSTA 2014}
\end{itemize}

\resheading{Presentations}
\begin{itemize}
    \resitem{ASE 2013 Paper Presentation and Tool Demonstration}
    \resitem{CESCA Day Poster Presentation}
\end{itemize}

\resheading{Research Experience}
%\begin{itemize}
%\item
  \ressubheading{Department of ECE, Virginia Tech}{Blacksburg, VA}{Undergraduate Research Assisant (NSF REU)}{2012 - Present}
    \begin{itemize}
      \resitem{Fall 2013}
        \begin{itemize}
          \resitem{Designed and implemented a supplemental static analysis
            algorithm to reduce the testing of redundant interleavings
          in dynamic systematic concurrent program analysis.}
      \end{itemize}

      \resitem{Summer 2013}
        \begin{itemize}
          \resitem{Implemented the HaPSet algorithm in C++ for concurrent
          programs using the POSIX threading library.}
          \resitem{Implemented, using C++, python and Intel PIN, a dynamic
          partial order reduction algorithm on a concurrent software testing
          system.}
          \resitem{Designed, implemented and tested a heuristic for state
          exploration reduction in dynamic concurrent software testing.}
        \end{itemize}

      \resitem{Spring 2013}
        \begin{itemize}
          \resitem{Implemented partial and higher order mutation operators on
            LLVM bitcode. Used the LLVM library for both static analysis and
          transformation. Operates on on C++11 and POSIX threading constructs.}
        \end{itemize}
          
      \resitem{Summer and Fall 2012}
        \begin{itemize}
          \resitem{Created a repository of over 60 real world concurrency bugs in
          large scale software and wrote simplified test cases for bugs using
          POSIX threads.}
        \end{itemize}
    \end{itemize}
%\end{itemize}

\resheading{School Projects}
\begin{description}
  \item[Embedded Systesm:]
    Designed an algorithm for an autonomous area mapping
    rover.
    \begin{itemize} 
        \resitem{The rover could be placed in a non-regular convex polyhedron and
        would generate a Cartesian map of its surroundings.}
        \resitem{Designed and tested a communication protocol to be used
        between a PIC and ARM processor.}
        \resitem{Implemented the wireless communication, sensor, motor controller and shaft
        encoder interfaces on two 8-bit PIC processors.}
        \resitem{Verified correct operation of the PICs using code
        instrumentation and a logic analyzer to create timing diagrams.}
    \end{itemize}
  \item[Digital Design II:] Realized synchronous/asynchronous circuits on an
    FPGA.
    \begin{itemize}
      \resitem{Used ModelSim+Quartus to simulate and synthesize verilog code}
      \resitem{Implemented modules to properly send data using SPI protocol}
      \resitem{Implemented modules to properly decode a modulated signal from an
        IR remote control}
      \resitem{Implemented modules to de-bounce electro-mechanical relays}
    \end{itemize}

  \item[AMP Lab Mentor:] Worked with two students on an independent project
    \begin{itemize}
        \resitem{Met weekly with two students working on an out-of-class project at
        Virginia Tech's Autonomous Mastery Prototyping Lab (AMP Lab)}
      \resitem {Their project was to create a wirelessly controlled robot capable
        of streaming video and audio to a remote user}
      \resitem{Gave guidance and advice on high level design choices and equipment
        choices}
      \resitem{Gave instructional demos on debugging embedded systems by using logic analyzers and code instrumentation}
  \end{itemize}

  \item[Other Projects:] Projects from various classes
    \begin{itemize}
      \resitem{Soldered surface mount and through-hole parts to create a digital
      resistance meter}
      \resitem{Implemented MIPS arithmetic instructions in verilog}
      \resitem{Created multi-threaded GUI applications using Qt}
      \resitem{Created a basic oscilloscope using a PIC32 micro-controller}
      \resitem{Designed, optimized and implemented boolean logic formulas for
      synchronous and asynchronous digital systems}
    \end{itemize}

\end{description}

\resheading{Skills}
\begin{description}
\item[Languages:]
C, C++11,\LaTeX, LLVM-IR, Bash, Python, Verilog, Haskell, Scheme
\item[Applications and Libraries:]
    LLVM, POSIX threads, Intel PIN, Autotools, GCC, GDB, git, svn, mercurial, cvs, Qt, ModelSim+Quartus
%\item[School Projects:]
    %Designed and implemented an autonomous rover capable of mapping an
    %enclosure using two PIC processors, one ARM processor, and multiple
    %sensors.  Programmed using SPI, $I^2C$, CAN and UART buses.  Implemented
    %multithreaded GUI applications using Qt. Realized synchronous/asynchronous
    %circuits on FPGAs and PLDs. 
\end{description}

\resheading{Interests}
\begin{description}
\item[Academic:] Compiler design, Functional Programming
\item[Computers:] Currently maintain Gentoo Linux systems, enjoy using and
		    learning Linux systems, and writing C software.
\item[Memberships:] Virginia Tech Linux and Unix Users Group, IEEE Robotics, Virginia Tech AMP Lab Mentor
\end{description}

\end{document}
