% resume.tex
%
% (c) 2002 Matthew Boedicker <mboedick@mboedick.org> (original author) http://mboedick.org
% (c) 2003-2007 David J. Grant <davidgrant-at-gmail.com> http://www.davidgrant.ca
%
% This work is licensed under the Creative Commons Attribution-ShareAlike 3.0 Unported License. To view a copy of this license, visit http://creativecommons.org/licenses/by-sa/3.0/ or send a letter to Creative Commons, 171 Second Street, Suite 300, San Francisco, California, 94105, USA.

\documentclass[letterpaper,11pt]{article}

%-----------------------------------------------------------
%Margin setup

\setlength{\voffset}{0.1in}
\setlength{\paperwidth}{8.5in}
\setlength{\paperheight}{11in}
\setlength{\headheight}{0in}
\setlength{\headsep}{0in}
\setlength{\textheight}{10.5in}
%\setlength{\textheight}{9.5in}
\setlength{\topmargin}{-0.25in}
\setlength{\textwidth}{7in}
\setlength{\topskip}{0in}
\setlength{\oddsidemargin}{-0.25in}
\setlength{\evensidemargin}{-0.25in}
%\setlength{\oddsidemargin}{0.5in}
%\setlength{\evensidemargin}{0.5in}
%-----------------------------------------------------------
%\usepackage{fullpage}
\usepackage{shading}
%\textheight=9.0in
\pagestyle{empty}
\raggedbottom
\raggedright
\setlength{\tabcolsep}{0in}

%-----------------------------------------------------------
%Custom commands
\newcommand{\resitem}[1]{\item #1 \vspace{-2pt}}
\newcommand{\resheading}[1]{{\large \parashade[.9]{sharpcorners}{\textbf{#1 \vphantom{p\^{E}}}}}}
\newcommand{\ressubheading}[4]{
\begin{tabular*}{6.5in}{l@{\extracolsep{\fill}}r}
		\textbf{#1} & #2 \\
		\textit{#3} & \textit{#4} \\
\end{tabular*}\vspace{-6pt}}
%-----------------------------------------------------------


\begin{document}

\begin{tabular*}{7in}{l@{\extracolsep{\fill}}r}
    \textbf{\Large Markus Kusano}  & 571-212-9696\\
    13402 Alfred Mill Ct. &  mukusano@vt.edu \\
    Herndon, VA 20171 & http://github.com/markus-kusano \\
\end{tabular*}
\\

\vspace{0.1in}


\resheading{Objective}
Acceptance to Virginia Tech Computer Engineering PhD program

\resheading{Education}
\begin{itemize}
\item
	\ressubheading{Virginia Tech}{Blacksburg, VA}{Computer Engineering (3.89/4.0 (Deans List with Distinction))}{Aug. 2009 - 2014}
	\begin{itemize}
            \resitem{Relevant courses: Embedded Systems Design, Applied Software Design, Digital Design II, Semiconductor Device Fundamentals, Electronics}
	\end{itemize}

\end{itemize}

\resheading{Research Experience}
\begin{itemize}
\item
    \ressubheading{Center for Embedded Systems for Secure Applications}{Blacksburg, VA}{Undergraduate Research Assisant}{2012 - Present}
    \begin{itemize}
        \resitem{Implemented partial and higher order mutation operators on
          LLVM bitcode. Focus on C++11 and POSIX threading constructs.
          Published tool paper at ASE'13\footnote{``CCmutator: A mutation
            generator for concurrency constructs in multithreaded C/C++
            applications," Markus Kusano and Chao Wang. IEEE/ACM International
          Conference on Automated Software Engineering.}}
        \resitem{Designed, implemented and tested a heuristic for state exploration reduction in dynamic concurrent software testing. Planned submission for FSE'13}
        \resitem{Implemented dynamic partial order reduction algorithm on a concurrent software testing system}
	\resitem{Created a repository of over 60 real world concurrency bugs in large scale software and wrote simplified test cases for bugs using POSIX threads.}
    \end{itemize}
\end{itemize}

\resheading{Skills}

\begin{description}
\item[Languages:]
C, C++11,\LaTeX, LLVM-IR, Bash, Haskell, Scheme, Verilog
\item[Applications and Libraries:]
    LLVM, POSIX threads, Autotools, GCC, GDB, git/svn/mercurial/cvs, Qt, ModelSim+Quartus
\item[School Projects:]
    Designed and implemented an autonomous rover capable of mapping an
    enclosure using two PIC processors, one ARM processor, and multiple
    sensors.  Programmed using SPI, $I^2C$, CAN and UART buses.  Implemented
    multithreaded GUI applications using Qt. Realized synchronous/asynchronous
    circuits on FPGAs and PLDs. 
\end{description}

\resheading{Interests}

\begin{description}
\item[Academic:] Compiler design, Functional Programming
\item[Computers:] Currently maintain Gentoo Linux systems, enjoy using and
		    learning Linux systems, and writing C software.
\item[Memberships:] Virginia Tech Linux and Unix Users Group, IEEE Robotics, Virginia Tech AMP Lab Mentor
\end{description}

\end{document}
